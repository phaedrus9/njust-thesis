\chapter{工作一}\label{chap:2}
%%%%%%%%%%%%%%%%%%%%%%%%%%%%%%%%%%%%%%%%%%%%%%%%%%%%%%%%%%%%%%%%%%%%%%%%%%%%%%%%%
\section{引言}\label{sec:201}
对于学位论文而言,\LaTeX{}又是体现在模块化处理、公式、图标、交叉引用等方面。模块化处理即将整个文稿切割成多个简单的子模块,然后利用主文件将文稿的子模块链接成一篇完整的文章(和编程语言中模块化、以及商业软件LsDYNA$^{\circledR}$中使用的include命令相同)。另外,排版系统中的格式定义系统也可以单独的模块化,由类文件(.cls)通过命令定义文中需要的版式等格式函数命令,通过格式文件(.sty)包含一些常用的包(package)。如此,文章中的格式信息和文稿中的内容就形成了相对独立的系统。对普通用户而言,只需要书写文稿内容,而将格式信息交由专业排版方进行(如图书馆、出版社等)。公式和图表的优势体现在,格式自动化对齐(相信使用MS的都有过公式窜行和表格窜页的感触)、交叉引用自动编号。

%%%%%%%%%%%%%%%%%%%%%%%%%%%%%%%%%%%%%%%%%%%%%%%%%%%%%%%%%%%%%%%%%%%%%%%%%%%%%%%%%
\section{本章一级标题}\label{sec:202}
\subsection{本章二级标题}\label{sec:2021}

ImageNet\cite{deng2009imagenet}

AlexNet\cite{krizhevsky2012imagenet}

%%%%%%%%%%%%%%%%%%%%%%%%%%%%%%%%%%%%%%%%%%%%%%%%%%%%%%%%%%%%%%%%%%%%%%%%%%%%%%%%%
\section{文档结构}\label{sec:2022}
