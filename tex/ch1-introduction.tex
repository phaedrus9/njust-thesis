\chapter{绪论}\label{chap:Introduction}
%%%%%%%%%%%%%%%%%%%%%%%%%%%%%%%%%%%%%%%%%%%%%%%%%%%%%%%%%%%%%%%%%%%%%%%%%%%%%%%%%
\section{研究的背景和意义}\label{sec:Background}
\LaTeX{}是一种基于\TeX{}的排版系统,主要利用命令行代码的形式对文稿进行格式化处理。相对于常用的可视化工具(如MS Word$^{\circledR}$)而言,其能够让作者更加专注于文章本身内容,而较多地将排版等重复任务交给编译系统,尤其是数学公式、参考文献或图标较多的科技文献。针对中文,\LaTeX{}提供有CTeX套装,并且国内较多院校都提供有\LaTeX{}格式的学位论文模板,中文期刊的排版系统中应用也较为广泛。

%%%%%%%%%%%%%%%%%%%%%%%%%%%%%%%%%%%%%%%%%%%%%%%%%%%%%%%%%%%%%%%%%%%%%%%%%%%%%%%%%
\section{国内外研究现状}\label{sec:ResStatus}

%%%%%%%%%%%%%%%%%%%%%%%%%%%%%%%%%%%%%%%%%%%%%%%%%%%%%%%%%%%%%%%%%%%%%%%%%%%%%%%%%
\section{研究概述及主要贡献}\label{sec:MainContribution}

%%%%%%%%%%%%%%%%%%%%%%%%%%%%%%%%%%%%%%%%%%%%%%%%%%%%%%%%%%%%%%%%%%%%%%%%%%%%%%%%%
\section{论文结构}\label{sec:PaperStructure}
本文组织结构安排如下:

第一章...

第二章...

第三章...

第四章...

第五章对于全文工作进行总结,并展望未来可行的研究方向。

